\documentclass[sigconf]{acmart}

\usepackage{hyperref}

\usepackage{endfloat}
\renewcommand{\efloatseparator}{\mbox{}} % no new page between figures

\usepackage{booktabs} % For formal tables

\settopmatter{printacmref=false} % Removes citation information below abstract
\renewcommand\footnotetextcopyrightpermission[1]{} % removes footnote with conference information in first column
\pagestyle{plain} % removes running headers

\begin{document}
\title{Big data analysis in Finance Sector}


\author{Dhanya Mathew}
\orcid{HID328}
\affiliation{%
  \institution{Indiana University}
  \streetaddress{711 N Park Ave}
  \city{Bloomington} 
  \state{Indiana} 
  \postcode{47408}
}
\email{dhmathew@iu.edu}

% The default list of authors is too long for headers}
\renewcommand{\shortauthors}{B. Trovato et al.}


\begin{abstract}
In order to understand what drives customer profit, we want to be able to predict what profit group (extremely unprofitable, average, extremely profitable etc.) a set of customers falls into based on their data at any given time.
\end{abstract}

\keywords{Random Forest, R, standard deviation}


\maketitle

\section{Introduction}

Big data as it's name implies, refers to large and complex data which continues to grow enormously day by day. There are huge number of sectors or applications including government, business, technology, universities, health-care, finance, manufacturing etc who make use of big data by obtaining meaningful information using big data technologies. This paper investigates how big data is helpful in financial firms in terms of predictive analysis and profitable growth. The finance sector is generating huge amounts of data on a daily basis from products and marketing, banking, business, to share market. Finance is a very sensitive field and any useful insight can make a positive impact on the overall turnover. Historic data analysis and real time data analysis are equally important in terms of finance sector. The key idea behind is how to retrieve the "signal" of relevant information from the bulk of data. Let us explore the wide range of possibilities of big data analysis that finance sector can come up with including decision making, discovery of new business opportunities, enhanced productivity and efficiency, risk management, fraud detection, innovation possibilities, efficiency and growth and a detailed view of customer segmentation in banking sector. 


\begin{acks}

  The authors would like to thank 

\end{acks}

\bibliographystyle{ACM-Reference-Format}
\bibliography{report} 

\end{document}
