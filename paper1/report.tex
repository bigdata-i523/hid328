\documentclass[sigconf]{acmart}

\usepackage{hyperref}

\usepackage{endfloat}
\renewcommand{\efloatseparator}{\mbox{}} % no new page between figures

\usepackage{booktabs} % For formal tables

\settopmatter{printacmref=false} % Removes citation information below abstract
\renewcommand\footnotetextcopyrightpermission[1]{} % removes footnote with conference information in first column
\pagestyle{plain} % removes running headers

\begin{document}
\title{Big data analysis in Finance Sector}


\author{Dhanya Mathew}
\orcid{HID328}
\affiliation{%
  \institution{Indiana University}
  \streetaddress{711 N Park Ave}
  \city{Bloomington} 
  \state{Indiana} 
  \postcode{47408}
}
\email{dhmathew@iu.edu}

% The default list of authors is too long for headers}
\renewcommand{\shortauthors}{B. Trovato et al.}


\begin{abstract}

In order to understand what drives an organization or company profit, we want to be able to predict the business trends, challenges, opportunities, risks and what profit group (extremely unprofitable, average, extremely profitable etc.) a set of customers falls into based on their data at any given time.\cite{https://www-935.ibm.com}

\end{abstract}

\keywords{i523, HID328, data-driven, data lakes, Hadoop, Random Forest}


\maketitle

\section{Introduction}

Big data as it's name implies, refers to large and complex data which continues to grow enormously day by day. There are huge number of sectors or applications including government, business, technology, universities, health-care, finance, manufacturing etc who make use of big data by obtaining meaningful information using big data technologies. This paper investigates how big data is helpful in financial firms in terms of predictive analysis and profitable growth. The finance sector is generating huge amounts of data on a daily basis from products and marketing, banking, business, to share market. Finance is a very sensitive field and any useful insight can make a positive impact on the overall turnover. Historic data analysis and real time data analysis are equally important in terms of finance sector. The key idea behind is how to retrieve the "signal" of relevant information form the bulk of data. Let us explore the wide range of possibilities of big data analysis that finance sector can come up with including decision making, discovery of new business opportunities, enhanced productivity and efficiency, risk management, fraud detection, innovation possibilities, efficiency and growth and a detailed view of customer segmentation in banking sector. \cite{}

\subsection{Efficient decision making}

The era of big data helps financial firms to take quality business decisions related to expanding revenues, managing costs, hiring resources etc based on effective data analysis which provide access to real-time insights.  Data-driven decision making is one of the key advantages of big data technologies. Data driven decision making approach includes data storage, data elaboration, data analysis and decision making. 

\textit{Data storage:} Even though big data does not defines by the size alone, we need the right means to store the huge volume and variety of data. Big data is distributed - stored across many machines and managed with Hadoop File System and distributed DBs like HBase and Apache Cassandra.

\textit{Data elaboration:} Generate combined information by eliminating unwanted data using data cleansing methods like grouping, joining, filtering etc(Spark, R, MapReduce, Storm). 

\textit{Data Analysis:} Big data analysis is the process of analyzing the data to derive the semantics of the available data to understand the hidden patterns, correlations, market trends, customer preferences which helps the organizations to take more informed decisions. Visualization tools include- Tableau,Google chart, D3, Fusion chart etc.

\textit{Decision making:} Data-driven decision making based on the analysis.

\subsection{Increased productivity and growth}
Compared to traditional data warehouses, the big data concept of Data lakes to store raw data offers more flexibility in data access and analysis. Large volumes of data are stored, managed and analyzed in data lakes  by using automated and sophisticated analytical tools. 

Data Lakes can be accessed by Machine learning algorithms, In-memory technologies, fast access DBs, big data queries and real-time analysis methods which consume less time to come up with meaningful information and reports.

\textit{Data lakes:} Data Lakes can be compared to the actual lakes where water doesn't get filled like that instead there are rivers or streams that bring water to it. In data lakes this is called ingestion of data. we collect all the data that we required to analyze to reach our goal irrespective of the source. These ‘streams’ of data come in several formats: structured data (simply said, data from a traditional relational database or even spreadsheet: rows and columns), unstructured data (social, video, email, text,…), data from all sorts of logs (e.g weblogs, clickstream analysis,…), XML, machine-to-machine, IoT and sensor data,, you name it (logs and XML are also called semi-structured data). There can be data filters in place based on the requirements.



\subsection{Identify business priorities}
As per the six sigma methodology, the three steps for aligning projects to business priorities based on gross profits are, 

    * Identify the relative importance of strategic business objectives
    * Identify the relative importance of specific key business processes
    * Calculate the relative importance of key metrics of key processes
    
There are different methodologies for identifying and prioritizing use cases for business priorities and innovations. The data-driven approach to find use cases includes BARC "Smart Data Science" methodology which apply techniques like pattern recognition and business intelligence. Industry communities are moving from traditional BI (data warehousing, latency, data sampling) to big data technologies
    
\subsection{Risk Management}
Financial firms especially banking sector are facing new regulatory requirements and challenges or risks each year. Big data adoption provide organizations a simplified and data-driven solution to mitigate the risks and helps to convert the data into usable information for regulatory reporting. Using data lakes and stronger analytic tools   also helps to foresee the expected impact quickly.

\subsection{Understand new business opportunities}
Big data will fundamentally change the way businesses
compete and operate. Companies that invest in and
successfully derive value from their data will have a distinct advantage over their competitors — a performance gap that will continue to grow as more relevant data is generated, emerging technologies and digital channels offer better acquisition and delivery mechanisms, and the technologies that enable faster, easier data analysis continue to develop. It is difficult to identify whats most important in the data, which technologies best suits the needs, who the customers are and what they expect. Being more data-driven gives an edge over competitors.  

Big data is the intersection of business strategy and data science, offering new opportunities to create competitive advantages. It allows companies to use data as a strategic asset, equipping them with pertinent real-time information when making decisions in order to eliminate inefficient operating processes, enhance the customer experience, take advantage of new markets, etc.
For many companies and businesses, big data is already a critical path to develop new products, services and business models

\subsection{Discovery of innovation possibilities}
Data is increasingly becoming a key differentiator between wildly profitable and struggling businesses. Exploring and analyzing data translates information into insight and drives to innovations. 

Firms are supposed to make decisions based on facts and data rather than intuition and should keep an open mind to innovation concepts. 

\subsection{Fraud detection}
One of the best ways to fight cybercrime is with early detection. Banks are prime targets for cybercriminals and fraudsters, and any kind of public breach creates a lot of embarrassment, bad publicity, and unwanted scrutiny. Clearly banks have a vested interest in any technology to identify and prevent a data breach or fraud.

Banks and financial services firms use analytics to differentiate fraudulent interactions from legitimate business transactions. By applying analytics and machine learning, they are able to define normal activity based on a customer's history and distinguish it from unusual behavior indicating fraud. The analysis systems suggest immediate actions, such as blocking irregular transactions, which stops fraud before it occurs and improves profitability.

\subsection{Cost effective information gathering}

Unlike traditional business intelligence systems, new techniques and technologies used with Big Data allow to gain useful information at a much lower cost. New architectures and the move from data silos to “data lakes” can provide substantial cost advantages, due in part
to greater scalability but also due to flexibility in the data analysis. In fact having all data sources in a data lake allows users to pull new reports on relatively new data, while in traditional data warehouses (DWHs) users have to extract, transform and load (ETL) new data into a static data model, which is expensive and costly from a time perspective. By using automated and sophisticated analytical tools that can store and analyze data faster and more easily, CFOs can reduce the overall cost to serve in relation to data elaboration.

Big Data adoption helps organizations
simplify and reduce the costs of taking
data from the source and converting it
into useable information for regulatory
reporting, including such data-intensive
activities as real-time simulations and
scenario analysis that are often required
by the regulator.

\subsection{Customer Segmentation and personalized marketing}

Banks have been under pressure to change from product-centric to customer-centric businesses. One way to achieve that transformation is  to better understand their customers through segmentation. Big data enables them to  group customers into distinct segments, which are defined by data sets that may include customer demographics, daily transactions, interactions with online and telephone customer service systems, and external data, such as the value of their homes. Promotions and marketing campaigns are then targeted to customers according to their  segments.

There are many segmentation identification algorithms available in the Big Data world.  Random Forest is one of the prominant algorithm. Apache spark, R are some of the technologies that have good integration with segmentation algorithms

\textit{Personalized Marketing:} One step beyond segment-based marketing is personalized marketing, which targets customers based on understanding of their individual buying habits. While it’s  supported by big data analysis of merchant records, financial services firms can also incorporate unstructured data from their customers' social media profiles in order to create a fuller picture of the customers' needs through customer sentiment analysis. Once those needs are understood, big data analysis can create a credit risk assessment in order to decide whether or not to go ahead with a transaction.

\subsection{Risks and considerations}

Big data plays an increasingly important role in the financial services sector, where it’s used for everything from targeting advertisements to optimizing portfolios. While these technologies have many benefits, critics are quick to point out that they can also become a source of discrimination if they're developed and/or used in an improper way.

\textit{Data Security:} This risk is obvious and often uppermost in our minds when we are considering the logistics of data collection and analysis. Data theft is a rampant and growing area of crime – and attacks are getting bigger and more damaging. 

\textit{Data Privacy:} Closely related to the issue of security is privacy. But in addition to ensuring that people’s personal data are safe from criminals, you need to be sure that the sensitive information you are storing and collecting isn’t going to be divulged through less malevolent but equally damaging misuse by yourself or by people to whom you have delegated responsibility for analyzing and reporting on it.

\textit{Bad Analytics:} Aka “getting it wrong.” Misinterpreting the patterns shown by your data and drawing causal links where there is in fact merely random coincidence is an obvious pitfall. Sales data may show a rise following a major sporting event, prompting you to draw a link between sports fans and your products or services, when in fact the rise is based on there being more people in town, and the rise would be equally dramatic after a large live music event.

\textit{Bad Data:} There might be situations where many data projects that start off on the wrong foot by collecting irrelevant, out of date, or erroneous data. This usually comes down to insufficient time being spent on designing the project strategy.


\section{Conclusion}
The Big Data revolution, however, offers new opportunities
for profitable growth, and financial services firms are responding enthusiastically. 

Apart from financial firms big data, continue to bring big benefits to the day to day life: advertisements focused on what you actually want to buy, smart cars that can help you avoid collisions, wearable or implantable devices that can monitor your health and notify your doctor if something is going wrong. 


\bibliographystyle{ACM-Reference-Format}
\bibliography{report} 

\end{document}
